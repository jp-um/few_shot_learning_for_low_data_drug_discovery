\begin{abstract}
	The discovery of new leads through ligand-based virtual screening in drug discovery is essentially a low-data problem, as data acquisition is both difficult and expensive to acquire. The application of conventional machine learning techniques to this problem domain is hindered by the requirement for large amounts of training data. In this work, we explore few-shot machine learning for LBVS, in which we build on the state-of-the-art, and introduce two new metric-based meta-learning techniques, Prototypical and Relation Networks, to this problem domain. We also explore the use of different embeddings and find that learned graph embeddings consistently perform better than extended-connectivity fingerprints. We conclude that the effectiveness of few-shot learning is highly dependent on the nature of the data. Few-shot learning models struggle to perform consistently on MUV and DUD-E data, in which the active compounds are structurally distinct. However, on Tox21 data, the few-shot models perform well, and we find that Prototypical Networks outperform the state of the art. Additionally, training these networks is substantially faster (up to 190\%) and therefore take a fraction of the time to train for comparable, or better, results.
\end{abstract}