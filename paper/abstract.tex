\begin{abstract}
	The discovery of new leads through ligand-based virtual screening in drug discovery is essentially a low-data problem, as data acquisition is both difficult and expensive. The requirement for large amounts of training data hinders the application of conventional machine learning techniques to this problem domain. This work explores few-shot machine learning for lead optimisation and hit identification. We build on the state-of-the-art and introduce two new metric-based meta-learning techniques, Prototypical and Relation Networks, to this problem domain. We also explore using different embeddings, namely extended-connectivity fingerprints (ECFP) and embeddings generated through graph convolutional networks (GCN), as inputs to neural networks for classification. This study shows that learned embeddings through GCNs consistently perform better than extended-connectivity fingerprints for toxicity and LBVS experiments. We conclude that the effectiveness of few-shot learning is highly dependent on the nature of the data. Few-shot learning models struggle to perform consistently on MUV and DUD-E data, in which the active compounds are structurally distinct. However, on Tox21 data, the few-shot models perform well, and we find that Prototypical Networks outperform the state-of-the-art based on the Matching Networks architecture. Additionally, training these networks is substantially faster (up to 190\%) and therefore takes a fraction of the time to train for comparable, or better, results.
\end{abstract}
