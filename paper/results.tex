\section{Results}

\subsection{Tox21}

In line with the results reported by \citet{altae2017low}, the few-shot learning models on Tox21 outperform the benchmark models significatly. The Matching Networks with IterRefLSTM performs well and obtain the best ROC results in a number of experiments. The fact that the same implementation for the Matching Networks (MNs) obtained slightly better results (1-14\% across the five support set experiments) than the state-of-the-art work, can be attributed to the set of atom descriptors used for the initial graph representations presented earlier. Our few-shot learning architecture implementation is identical to the work of \citet{altae2017low}, however variations in how the model learns could be present. Hence, we focus mainly on the performance of how our implementations performed against each other. The results from the Prototypical Networks (PNs) overall outperform the results from the MNs based on statistical analysis (see Table \ref{table:Tox21-mean}). Meanwhile, MNs and PNs, overall outperform Relation Networks (RNs) in both ROC and PRC performance. 

Results for one shot-learning do not provide a clear-cut choice between our implementations for MNs and PNs with the IterRefLSTM, which is expected due to the architecture of these methods.  In a one-shot learning scenario, MNs and PNs are conceptually similar. The main difference lies in the distance function used as for MNs we use the cosine distance, while for PNs, we make use of the euclidean distance, as proposed in the original literature which introduced these two techniques. They both achieve comparable performance on Tox21 targets for one-shot learning. The performance of MNs for this scenario is consistent with the state-of-the-art work and for such a difficult scenario (i.e. learning with only one example from each class), results are promising. The \textit{prototypes} in PNs are a mean of all embeddings for each class in the support set. The euclidean distance between the \textit{prototypes} and each embedding from the query set is calculated to predict the query's activity. As in one-shot learning we only have one example per class, the \textit{prototypes} are equivalent to the embedding for each class, making this identical to the MNs. 

\begin{table*}
	\caption{Mean ROC and PRC Scores with standard deviation for ML Models for the Tox21 Test Targets over 20 rounds of testing. Bold text illustrates the best obtained value. The first column shows the composition of the support set. The reproduced results from the model used in \citet{altae2017low} is the MatchingNet.}
	\centering
	% \ra{1.3}
	% {\renewcommand{\arraystretch}{1}}
	\resizebox{\textwidth}{!}{%
		\begin{tabular}{@{}cccccccc@{}}
			\hline
			\textbf{Tox21} & \textbf{Metric} & \textbf{RF} & \textbf{Graph Conv} & \textbf{SiameseNet} & \textbf{MatchingNet} & \textbf{ProtoNet} & \textbf{RelationNet} \\
			\hline
			10+/10- & ROC & 0.617 ± 0.060 & 0.620 ± 0.065 & 0.825 ± 0.043 & 0.824 ± 0.022 & \textbf{0.826 ± 0.034} & 0.814 ± 0.030 \\ & PRC & 0.158 ± 0.102 & 0.150 ± 0.095 & 0.226 ± 0.107 & 0.367 ± 0.105 & \textbf{0.384 ± 0.105} & 0.360 ± 0.102\\
			\hline
			5+/10- & ROC & 0.602 ± 0.059 & 0.610 ± 0.062 & 0.828 ± 0.069 & \textbf{0.824 ± 0.033} & 0.823 ± 0.038 & 0.822 ± 0.023 \\ & PRC & 0.148 ± 0.090 & 0.152 ± 0.094 & 0.190 ± 0.094 & 0.369 ± 0.110 & \textbf{0.388 ± 0.111} & 0.355 ± 0.104\\
			\hline
			1+/10- & ROC & 0.563 ± 0.068 & 0.558 ± 0.076 & 0.836 ± 0.138 & 0.822 ± 0.025 & \textbf{0.826 ± 0.032} & 0.814 ± 0.028 \\ & PRC & 0.128 ± 0.084 & 0.126 ± 0.075 & 0.099 ± 0.093 & 0.301 ± 0.103 & \textbf{0.384 ± 0.096} & 0.325 ± 0.103\\
			\hline
			1+/5- & ROC & 0.534 ± 0.066 & 0.559 ± 0.090 & 0.807 ± 0.159 & \textbf{0.820 ± 0.033} & \textbf{0.820 ± 0.033} & 0.819 ± 0.023 \\ & PRC & 0.112 ± 0.059 & 0.128 ± 0.080 & 0.106 ± 0.086 & 0.339 ± 0.115 & \textbf{0.362 ± 0.106} & 0.318 ± 0.108\\
			\hline
			1+/1- & ROC & 0.550 ± 0.061 & 0.548 ± 0.102 & 0.818 ± 0.075 & 0.819 ± 0.036 & \textbf{0.820 ± 0.030} & 0.813 ± 0.029 \\ & PRC & 0.118 ± 0.068 & 0.123 ± 0.082 & 0.198 ± 0.102 & 0.352 ± 0.121 & \textbf{0.373 ± 0.102} & 0.342 ± 0.093\\
			\hline
		\end{tabular}
	}
	\label{table:Tox21-mean}
\end{table*}

\subsection{MUV}

Each active in the MUV dataset is structurally distinct from the other, making each data sample maximally informative. Therefore, structural similarities cannot be exploited on unseen active molecules. The baseline benchmark tests consistently outperformed few-shot learning techniques. \citet{altae2017low} report that the results obtained through the GCNs baseline also struggle in performance, however, from our tests and statistical analysis we find that this is not the case for all MUV targets. For most targets, there is no significant difference between the scores obtained through the RFs and GCNs baselines. RNs obtain the best ROC scores in one instance on the MUV-832 target when trained with a 1+/10- support set, obtaining a mean ROC-AUC score of $0.683 \pm 0.010$. However, this result is not consistent and the performance is only observed in this single instance. Other than this single instance, our results are consistent with the conclusion from the state-of-the-art that baseline machine learning outperforms few-shot machine learning techniques on the MUV dataset.

\begin{table*}
	\caption{Mean ROC and PRC Scores with standard deviation for ML Models for MUV Test Targets over 20 rounds of testing. Bold text illustrates the best obtained value. The first column shows the composition of the support set. The reproduced results from the model used in \citet{altae2017low} is the MatchingNet.}
	\centering
	\resizebox{\textwidth}{!}{%
		\begin{tabular}{@{}cccccccc@{}}
			\hline
			\textbf{MUV} & \textbf{Metric} & \textbf{RF} & \textbf{Graph Conv} & \textbf{SiameseNet} & \textbf{MatchingNet} & \textbf{ProtoNet} & \textbf{RelationNet} \\
			\hline
			10+/10- & ROC & \textbf{0.728 ± 0.145} & 0.713 ± 0.133 & 0.562 ± 0.046 & 0.628 ± 0.096 & 0.599 ± 0.085 & 0.490 ± 0.071 \\ & PRC & \textbf{0.066 ± 0.073} & 0.009 ± 0.012 & 0.001 ± 0.000 & 0.007 ± 0.010 & 0.003 ± 0.002 & 0.002 ± 0.001\\
			\hline
			5+/10- & ROC & \textbf{0.696 ± 0.132} & 0.666 ± 0.115 & 0.550 ± 0.054 & 0.516 ± 0.085 & 0.576 ± 0.055 & 0.502 ± 0.072 \\ & PRC & \textbf{0.071 ± 0.076} & 0.015 ± 0.022 & 0.001 ± 0.001 & 0.003 ± 0.002 & 0.003 ± 0.002 & 0.003 ± 0.002\\
			\hline
			1+/10- & ROC & \textbf{0.599 ± 0.104} & 0.585 ± 0.116 & 0.648 ± 0.158 & 0.492 ± 0.082 & 0.540 ± 0.053 & 0.547 ± 0.090 \\ & PRC & \textbf{0.021 ± 0.032} & 0.006 ± 0.008 & 0.001 ± 0.002 & 0.002 ± 0.001 & 0.003 ± 0.002 & 0.003 ± 0.002\\
			\hline
			1+/5- & ROC & \textbf{0.587 ± 0.106} & 0.585 ± 0.126 & 0.613 ± 0.179 & 0.461 ± 0.046 & 0.494 ± 0.050 & 0.500 ± 0.000 \\ & PRC & \textbf{0.027 ± 0.040} & 0.006 ± 0.008 & 0.001 ± 0.002 & 0.002 ± 0.001 & 0.002 ± 0.001 & 0.002 ± 0.000\\
			\hline
			1+/1- & ROC & 0.573 ± 0.103 & \textbf{0.577 ± 0.147} & 0.620 ± 0.138 & 0.507 ± 0.037 & 0.505 ± 0.030 & 0.484 ± 0.060 \\ & PRC & \textbf{0.022 ± 0.036} & 0.006 ± 0.007 & 0.004 ± 0.011 & 0.002 ± 0.000 & 0.003 ± 0.001 & 0.002 ± 0.001\\
			\hline
		\end{tabular}
	}
	\label{table:muv-mean}
\end{table*}

\subsection{GPCR subset of the DUD-E}

For the ADRB2 target, the few-shot learning models achieve stellar performance based on ROC and PRC scores. The results are close to a perfect classifier, which raises concerns about the underlying data. Our hypothesis is that the underlying data contains an inherent bias, which is confirmed by further research on the matter. Some studies indicate that the DUD-E dataset has limited chemical space and bias from the decoy compound selection process \cite{smusz2013influence, wallach2018most}. \citet{chen2019hidden} investigate this further to establish the effect these characteristics have on CNN models. The authors conclude that there is analogue bias within the set of actives within the targets (intra-target analogue bias), and also between the actives of different targets (inter-target analogue bias). They also provide evidence that there is also bias in decoy selection through the selection criteria for decoys. Results obtained from the DUD-E dataset are not conclusive. On the other hand, for the decoys available for the CXCR4 target, the RF model excels and outperforms the few-shot learning models. Seeing that the GCN benchmark model also performed significantly better than few-shot learning models, this implies that there is a clear benefit of training on the same data from the target, as opposed to the few-shot learning models which are trained on other targets instead. Having such mixed results on two different targets within the same subset of the dataset does not give us a conclusive picture of whether few-shot learning is effective on this dataset.

\begin{table*}
	\caption{Mean ROC and PRC Scores with standard deviation for ML Models for DUD-E GPCR Test Targets over 20 rounds of testing. Bold text illustrates the best obtained value. The first column shows the composition of the support set.}
	\centering
	\resizebox{\textwidth}{!}{%
		\begin{tabular}{@{}cccccccc@{}}
			\hline
			\textbf{DUDE-GPCR} & \textbf{Metric} & \textbf{RF} & \textbf{Graph Conv} & \textbf{SiameseNet} & \textbf{MatchingNet} & \textbf{ProtoNet} & \textbf{RelationNet} \\
			\hline
			10+/10- & ROC & \textbf{0.982 ± 0.018} & 0.940 ± 0.039 & 0.784 ± 0.215 & 0.900 ± 0.102 & 0.816 ± 0.187 & 0.928 ± 0.008 \\ & PRC & \textbf{0.872 ± 0.102} & 0.504 ± 0.225 & 0.489 ± 0.475 & 0.535 ± 0.451 & 0.552 ± 0.445 & 0.562 ± 0.020\\
			\hline
			5+/10- & ROC & \textbf{0.958 ± 0.023} & 0.901 ± 0.058 & 0.761 ± 0.238 & 0.845 ± 0.153 & 0.843 ± 0.181 & 0.850 ± 0.149 \\ & PRC & \textbf{0.762 ± 0.119} & 0.428 ± 0.247 & 0.495 ± 0.465 & 0.506 ± 0.477 & 0.559 ± 0.439 & 0.523 ± 0.447\\
			\hline
			1+/10- & ROC & 0.854 ± 0.071 & 0.788 ± 0.098 & 0.759 ± 0.247 & \textbf{0.881 ± 0.119} & 0.841 ± 0.159 & 0.866 ± 0.132 \\ & PRC & 0.360 ± 0.136 & 0.230 ± 0.176 & 0.474 ± 0.445 & 0.521 ± 0.455 & 0.504 ± 0.463 & \textbf{0.541 ± 0.433}\\
			\hline
			1+/5- & ROC & \textbf{0.858 ± 0.084} & 0.763 ± 0.087 & 0.759 ± 0.246 & 0.851 ± 0.155 & 0.793 ± 0.211 & 0.848 ± 0.149 \\ & PRC & 0.378 ± 0.123 & 0.221 ± 0.153 & 0.482 ± 0.444 & 0.516 ± 0.438 & \textbf{0.519 ± 0.474} & 0.490 ± 0.427\\
			\hline
			1+/1- & ROC & 0.804 ± 0.108 & 0.710 ± 0.121 & 0.771 ± 0.228 & 0.795 ± 0.203 & \textbf{0.865 ± 0.133} & 0.747 ± 0.251 \\ & PRC & 0.301 ± 0.168 & 0.116 ± 0.121 & 0.500 ± 0.417 & 0.511 ± 0.470 & \textbf{0.543 ± 0.439} & 0.500 ± 0.465\\
			\hline
		\end{tabular}
	}
	\label{table:dude-mean}
\end{table*}

\subsection{Comparison with the State of the Art}

We tabulate the ROC results obtained by \citet{altae2017low} in Table \ref{table:Tox21-sota-ours} and compare them to the best results obtained from our implementations. The best network is selected using statistical analysis, comparing the results from 20 test rounds for each experiment across all the techniques employed. Instances in which the best networks contains more than one indicate that we do not find any statistically significant difference between the results obtained from that specific technique. Prototypical Network has the highest frequency of being identified as the best network on Tox21 data based on ROC scores. We remind the reader that while we also report the PR-AUC score from our experiments, this metric is not available in the study by \citet{altae2017low}, hence why the results reported in Table \ref{table:Tox21-sota-ours} contain only ROC results. We highlight that for the PRC metrics, Prototypical Networks consistently performed well, obtaining the best PRC scores throughout all Tox21 targets. Using statistical analysis, Matching Networks and Relation Networks also match the performance in some cases. The PRC is used to determine how well the model predicts active compounds, as it is the ratio of true positives divided by the sum of true positives and false positives. Therefore, we strongly believe that in machine learning experiments for virtual screening, this metric should be used in addition to ROC scores. We attribute any improvement in ROC for Matching Networks in our implementation over the state of the art to the featurisation of molecule and variability which might arise through machine learning. However, we reiterate that all results reported in this study are compared homogenously using the same machine learning architectures. 

\begin{table}[ht]
	\centering
	\begin{tabular}{@{}cccccc@{}}
	\textbf{Target} & \textbf{Support Set} & \textbf{SOTA} & \textbf{SOTA ROC} & \textbf{Obtained ROC} & \textbf{Best Networks} \\
	SR-HSE & 10+/10- & MN & 0.772 ± 0.002 & \textbf{0.793 ± 0.002} & MN \\
	SR-HSE & 5+/10- & MN & 0.771 ± 0.002 & \textbf{0.791 ± 0.003} & RN \\
	SR-HSE & 1+/10- & MN & 0.671 ± 0.007 & \textbf{0.788 ± 0.001} & MN \\
	SR-HSE & 1+/5- & MN & 0.729 ± 0.003 & \textbf{0.789 ± 0.001} & RN \\
	SR-HSE & 1+/1- & MN & 0.767 ± 0.001 & \textbf{0.779 ± 0.007} & PN \\
	SR-MMP & 10+/10- & MN & 0.838 ± 0.001 & \textbf{0.845 ± 0.015} & MN/PN/RN \\
	SR-MMP & 5+/10- & MN & 0.847 ± 0.001 & \textbf{0.853 ± 0.007} & MN/PN \\
	SR-MMP & 1+/10- & SN & 0.809 ± 0.020 & \textbf{0.849 ± 0.005} & PN \\
	SR-MMP & 1+/5- & MN & 0.799 ± 0.002 & \textbf{0.853 ± 0.001} & MN \\
	SR-MMP & 1+/1- & MN & 0.835 ± 0.001 & \textbf{0.851 ± 0.008} & MN \\
	SR-p53 & 10+/10- & MN & 0.823 ± 0.002 & \textbf{0.850 ± 0.004} & PN \\
	SR-p53 & 5+/10- & MN & 0.830 ± 0.001 & \textbf{0.852 ± 0.009} & PN \\
	SR-p53 & 1+/10- & SN & 0.726 ± 0.173 & \textbf{0.848 ± 0.005} & PN \\
	SR-p53 & 1+/5- & MN & 0.795 ± 0.005 & \textbf{0.840 ± 0.005} & PN \\
	SR-p53 & 1+/1- & MN & 0.827 ± 0.001 & \textbf{0.838 ± 0.004} & MN \\
	\end{tabular}
	\caption[Comparing our best ROC-AUC scores with the SOTA results on Tox21.]{Comparison of our best ROC-AUC scores against the state of the art (SOTA) results from \citet{altae2017low} on the Tox21 dataset. The best networks reported are based on the statistical analysis carried out in Section \ref{tox21results}. Values are mean values with standard deviation over 20 rounds of testing. Best values are highlighted in bold text.}
	\label{table:Tox21-sota-ours}
\end{table}

\subsection{ECFP vs Graph-Learned Embeddings}

We also ran an experiment to test whether the molecular representation affects the performance in few-shot learning. These experiments are run on the Tox21 dataset, using Prototypical Networks, as these performed consistently well in our other experiments. ECFPs are based on the topology and a number of atom descriptors, in which the molecule is fragmented into local neighbourhoods and hashed into a vector. On the other hand, graph-learned embeddings are guided by gradient descent during training to produce a more relevant latent space embedding for the molecule. A neural network was used to learn a differentiable molecular embedding, from the ECFP, of the same size (a vector of size 128) as the one produced by the GCN. The results obtained using a learned embedding from GCNs consistently outperform the ones in which an ECFP was used.

\subsection{Training Times}

From the results on the Tox21 dataset, MNs, PNs and RNs obtain good predictive performance, however, it is evident from the presented result that the two latter networks are much faster to train on the same hardware. From our experiments on the three Tox21 targets, MNs and PNs were the most consistent in results. As the decrease in training times is substantial, by over 150\% between MNs and both PNs and RNs, we believe that this puts the latter two networks at an advantage. Faster training times allow faster turnaround of results from datasets, while requiring less intense use of computer hardware. This increase in efficiency also allows scientists to perform a more rigorous hyperparameter search on various datasets in a shorter time.
